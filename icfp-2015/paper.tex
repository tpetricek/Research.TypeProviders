\documentclass[preprint]{sigplanconf}

\usepackage{xcolor}
\usepackage{graphicx}
\usepackage[fleqn]{amsmath}
\usepackage[hang,flushmargin]{footmisc} 
\usepackage{graphics} 
\usepackage{stmaryrd}
\usepackage{amsthm}
\usepackage{hyperref}

% =================================================================================================

\newcommand{\langl}{\begin{picture}(4.5,7)
\put(1.1,2.5){\rotatebox{60}{\line(1,0){5.5}}}
\put(1.1,2.5){\rotatebox{300}{\line(1,0){5.5}}}
\end{picture}}
\newcommand{\rangl}{\begin{picture}(4.5,7)
\put(.9,2.5){\rotatebox{120}{\line(1,0){5.5}}}
\put(.9,2.5){\rotatebox{240}{\line(1,0){5.5}}}
\end{picture}}

\newcommand{\lang}{\begin{picture}(5,7)
\put(1.1,2.5){\rotatebox{45}{\line(1,0){6.0}}}
\put(1.1,2.5){\rotatebox{315}{\line(1,0){6.0}}}
\end{picture}}
\newcommand{\rang}{\begin{picture}(5,7)
\put(.1,2.5){\rotatebox{135}{\line(1,0){6.0}}}
\put(.1,2.5){\rotatebox{225}{\line(1,0){6.0}}}
\end{picture}} 

\definecolor{cmtclr}{rgb}{0.0,0.6,0.0}
\definecolor{kvdclr}{rgb}{0.0,0.0,0.6}
\definecolor{strclr}{rgb}{0.5,0.1,0.0}
\definecolor{prepclr}{rgb}{0.0,0.0,0.0}

\newcommand{\sem}[1]{\llbracket #1 \rrbracket}
\newcommand{\ignp}{\_\kern-.5ex\_\kern-.5ex\_ }
\newcommand{\kvd}[1]{\textnormal{\textcolor{kvdclr}{\sffamily #1}}}
\newcommand{\str}[1]{\textnormal{\textcolor{strclr}{\sffamily "#1"}}}
\newcommand{\ident}[1]{\textnormal{\sffamily #1}}
\newcommand{\lident}[1]{\textnormal{\sffamily 
  \`{}\hspace{-0.25em}\`{}\hspace{-0.1em}#1\`{}\hspace{-0.25em}\`{}}}
\newcommand{\cmt}[1]{\textit{\sffamily\textcolor{cmtclr}{#1}}}

% =================================================================================================

\begin{document}

\special{papersize=8.5in,11in}
\setlength{\pdfpageheight}{\paperheight}
\setlength{\pdfpagewidth}{\paperwidth}
\conferenceinfo{CONF 'yy}{Month d--d, 20yy, City, ST, Country} 
\copyrightyear{20yy} 
\copyrightdata{978-1-nnnn-nnnn-n/yy/mm} 
\doi{nnnnnnn.nnnnnnn}

%\titlebanner{banner above paper title}        % These are ignored unless
%\preprintfooter{short description of paper}   % 'preprint' option specified.

\title{F\# Data: \textnormal{Making structured data first-class citizens}}
%\subtitle{Subtitle Text, if any}

\authorinfo{Tomas Petricek}
           {University of Cambridge}
           {tomas@tomasp.net}
%\authorinfo{Name2\and Name3}
%           {Affiliation2/3}
%           {Email2/3}

\maketitle

% -------------------------------------------------------------------------------------------------

\begin{abstract}
Most statically typed languages assume that programs run in a closed world, but this is not the 
case. Modern applications interact with external services and often access data in structured 
formats such as XML, CSV and JSON. Static type systems do not understand such external data sources
and only make data access more cumbersome. Should we just give up and leave the messy world of 
external data to dynamic typing and runtime checks?

Of course, we should not give up! In this paper, we show how to integrate external data sources 
into the F\# type system. As most real-world data on the web do not come with an explicit schema, 
we develop a type inference algorithm that infers a type from representative samples. Next, we use 
the type provider mechanism for integrating the inferred structures into the F\# type
system. 

The resulting library significantly reduces the amount of code that developers need to write when 
accessing data. It also provides additional safety guarantees -- arguably, as much as 
possible if we abandon the (incorrect) closed world assumption.
\end{abstract}

\category{D.3.3}{Programming Languages}{Language Constructs and Features }
\keywords F\#, Type Providers, JSON, XML, Data, Type Inference

% =================================================================================================

\section{Introduction}

\newpage

\footnote{\sffamily{http://api.openweathermap.org/data/2.5/weather?q=Prague\&units=metric}}

Bacon ipsum dolor amet biltong strip steak pancetta tenderloin. Picanha pork pig shank. Jowl ground round kielbasa capicola swine tenderloin tri-tip beef ribs brisket picanha boudin. Turkey ham hock jowl andouille hamburger pork chop shankle bacon, shank bresaola jerky chuck biltong sirloin meatball.

%
\begin{equation*}
\begin{array}{l}
 \kvd{let}~\ident{data}=\ident{Http.Request}(\str{http://weather.org/?q=Prague}) \\[0.25em]
 \kvd{match}~\ident{JsonValue.Parse}(\ident{data})~\kvd{with} \\[0.25em]
 |~\ident{Record}(\ident{root})\rightarrow \\[0.25em]
 \quad \kvd{match}~\ident{Map.find}~\str{main}~\ident{root}~\kvd{with} \\[0.25em]
 \quad |~\ident{Record}(\ident{main})\rightarrow \\[0.25em]
 \quad \quad \kvd{match}~\ident{Map.find}~\str{temp}~\ident{main}~\kvd{with} \\[0.25em]
 \quad \quad |~\ident{Number}(\ident{num})\rightarrow \ident{printfn}~\str{Lovely \%f degrees!}~\ident{num}
\end{array}
\end{equation*}
%


%
\begin{equation*}
\begin{array}{l}
 \kvd{type}~\ident{W} = \ident{JsonProvider}\langl\str{http://weather.org/?q=Prague}\rangl \\[0.25em]
 \ident{printfn}~\str{Lovely \%f degrees!}~(\ident{W.GetSample().Main.Temp})
\end{array}
\end{equation*}
%


The text of the paper begins here.

% -------------------------------------------------------------------------------------------------

\newpage
~
\newpage

\appendix
\section{Appendix Title}

This is the text of the appendix, if you need one.

\acks

Acknowledgments, if needed.

% We recommend abbrvnat bibliography style.

\bibliographystyle{abbrvnat}

% The bibliography should be embedded for final submission.

\begin{thebibliography}{}
\softraggedright

\bibitem[Smith et~al.(2009)Smith, Jones]{smith02}
P. Q. Smith, and X. Y. Jones. ...reference text...

\end{thebibliography}


\end{document}

%                       Revision History
%                       -------- -------
%  Date         Person  Ver.    Change
%  ----         ------  ----    ------

%  2013.06.29   TU      0.1--4  comments on permission/copyright notices

